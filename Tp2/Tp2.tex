% Options for packages loaded elsewhere
\PassOptionsToPackage{unicode}{hyperref}
\PassOptionsToPackage{hyphens}{url}
%
\documentclass[
]{article}
\usepackage{amsmath,amssymb}
\usepackage{lmodern}
\usepackage{iftex}
\ifPDFTeX
  \usepackage[T1]{fontenc}
  \usepackage[utf8]{inputenc}
  \usepackage{textcomp} % provide euro and other symbols
\else % if luatex or xetex
  \usepackage{unicode-math}
  \defaultfontfeatures{Scale=MatchLowercase}
  \defaultfontfeatures[\rmfamily]{Ligatures=TeX,Scale=1}
\fi
% Use upquote if available, for straight quotes in verbatim environments
\IfFileExists{upquote.sty}{\usepackage{upquote}}{}
\IfFileExists{microtype.sty}{% use microtype if available
  \usepackage[]{microtype}
  \UseMicrotypeSet[protrusion]{basicmath} % disable protrusion for tt fonts
}{}
\makeatletter
\@ifundefined{KOMAClassName}{% if non-KOMA class
  \IfFileExists{parskip.sty}{%
    \usepackage{parskip}
  }{% else
    \setlength{\parindent}{0pt}
    \setlength{\parskip}{6pt plus 2pt minus 1pt}}
}{% if KOMA class
  \KOMAoptions{parskip=half}}
\makeatother
\usepackage{xcolor}
\usepackage[margin=1in]{geometry}
\usepackage{color}
\usepackage{fancyvrb}
\newcommand{\VerbBar}{|}
\newcommand{\VERB}{\Verb[commandchars=\\\{\}]}
\DefineVerbatimEnvironment{Highlighting}{Verbatim}{commandchars=\\\{\}}
% Add ',fontsize=\small' for more characters per line
\usepackage{framed}
\definecolor{shadecolor}{RGB}{248,248,248}
\newenvironment{Shaded}{\begin{snugshade}}{\end{snugshade}}
\newcommand{\AlertTok}[1]{\textcolor[rgb]{0.94,0.16,0.16}{#1}}
\newcommand{\AnnotationTok}[1]{\textcolor[rgb]{0.56,0.35,0.01}{\textbf{\textit{#1}}}}
\newcommand{\AttributeTok}[1]{\textcolor[rgb]{0.77,0.63,0.00}{#1}}
\newcommand{\BaseNTok}[1]{\textcolor[rgb]{0.00,0.00,0.81}{#1}}
\newcommand{\BuiltInTok}[1]{#1}
\newcommand{\CharTok}[1]{\textcolor[rgb]{0.31,0.60,0.02}{#1}}
\newcommand{\CommentTok}[1]{\textcolor[rgb]{0.56,0.35,0.01}{\textit{#1}}}
\newcommand{\CommentVarTok}[1]{\textcolor[rgb]{0.56,0.35,0.01}{\textbf{\textit{#1}}}}
\newcommand{\ConstantTok}[1]{\textcolor[rgb]{0.00,0.00,0.00}{#1}}
\newcommand{\ControlFlowTok}[1]{\textcolor[rgb]{0.13,0.29,0.53}{\textbf{#1}}}
\newcommand{\DataTypeTok}[1]{\textcolor[rgb]{0.13,0.29,0.53}{#1}}
\newcommand{\DecValTok}[1]{\textcolor[rgb]{0.00,0.00,0.81}{#1}}
\newcommand{\DocumentationTok}[1]{\textcolor[rgb]{0.56,0.35,0.01}{\textbf{\textit{#1}}}}
\newcommand{\ErrorTok}[1]{\textcolor[rgb]{0.64,0.00,0.00}{\textbf{#1}}}
\newcommand{\ExtensionTok}[1]{#1}
\newcommand{\FloatTok}[1]{\textcolor[rgb]{0.00,0.00,0.81}{#1}}
\newcommand{\FunctionTok}[1]{\textcolor[rgb]{0.00,0.00,0.00}{#1}}
\newcommand{\ImportTok}[1]{#1}
\newcommand{\InformationTok}[1]{\textcolor[rgb]{0.56,0.35,0.01}{\textbf{\textit{#1}}}}
\newcommand{\KeywordTok}[1]{\textcolor[rgb]{0.13,0.29,0.53}{\textbf{#1}}}
\newcommand{\NormalTok}[1]{#1}
\newcommand{\OperatorTok}[1]{\textcolor[rgb]{0.81,0.36,0.00}{\textbf{#1}}}
\newcommand{\OtherTok}[1]{\textcolor[rgb]{0.56,0.35,0.01}{#1}}
\newcommand{\PreprocessorTok}[1]{\textcolor[rgb]{0.56,0.35,0.01}{\textit{#1}}}
\newcommand{\RegionMarkerTok}[1]{#1}
\newcommand{\SpecialCharTok}[1]{\textcolor[rgb]{0.00,0.00,0.00}{#1}}
\newcommand{\SpecialStringTok}[1]{\textcolor[rgb]{0.31,0.60,0.02}{#1}}
\newcommand{\StringTok}[1]{\textcolor[rgb]{0.31,0.60,0.02}{#1}}
\newcommand{\VariableTok}[1]{\textcolor[rgb]{0.00,0.00,0.00}{#1}}
\newcommand{\VerbatimStringTok}[1]{\textcolor[rgb]{0.31,0.60,0.02}{#1}}
\newcommand{\WarningTok}[1]{\textcolor[rgb]{0.56,0.35,0.01}{\textbf{\textit{#1}}}}
\usepackage{longtable,booktabs,array}
\usepackage{calc} % for calculating minipage widths
% Correct order of tables after \paragraph or \subparagraph
\usepackage{etoolbox}
\makeatletter
\patchcmd\longtable{\par}{\if@noskipsec\mbox{}\fi\par}{}{}
\makeatother
% Allow footnotes in longtable head/foot
\IfFileExists{footnotehyper.sty}{\usepackage{footnotehyper}}{\usepackage{footnote}}
\makesavenoteenv{longtable}
\usepackage{graphicx}
\makeatletter
\def\maxwidth{\ifdim\Gin@nat@width>\linewidth\linewidth\else\Gin@nat@width\fi}
\def\maxheight{\ifdim\Gin@nat@height>\textheight\textheight\else\Gin@nat@height\fi}
\makeatother
% Scale images if necessary, so that they will not overflow the page
% margins by default, and it is still possible to overwrite the defaults
% using explicit options in \includegraphics[width, height, ...]{}
\setkeys{Gin}{width=\maxwidth,height=\maxheight,keepaspectratio}
% Set default figure placement to htbp
\makeatletter
\def\fps@figure{htbp}
\makeatother
\setlength{\emergencystretch}{3em} % prevent overfull lines
\providecommand{\tightlist}{%
  \setlength{\itemsep}{0pt}\setlength{\parskip}{0pt}}
\setcounter{secnumdepth}{-\maxdimen} % remove section numbering
\ifLuaTeX
  \usepackage{selnolig}  % disable illegal ligatures
\fi
\IfFileExists{bookmark.sty}{\usepackage{bookmark}}{\usepackage{hyperref}}
\IfFileExists{xurl.sty}{\usepackage{xurl}}{} % add URL line breaks if available
\urlstyle{same} % disable monospaced font for URLs
\hypersetup{
  pdftitle={Proyecto I},
  hidelinks,
  pdfcreator={LaTeX via pandoc}}

\title{Proyecto I}
\author{}
\date{\vspace{-2.5em}2023-04-16}

\begin{document}
\maketitle

TP2- ITBA Relevamiento y análisis de Bases de datos

Profesoras:

Bouret, María Gabriela

Peralta Ramos, Angélica

Ian Dalton - 62345

Fecha de entrega: 17/4/23

\newpage

Ficha del dataset

\textbf{Nombre del dataset: }microdatos evyth

\textbf{Fuente:
}\href{https://datos.gob.ar/dataset/turismo-encuesta-viajes-turismo-hogares-evyth}{Ministerio
de Turismo y Deportes}

\textbf{Rango de fechas: }

Desde: 1/1/2012

Hasta: 31/12/2022

\textbf{Cantidad de filas: }434806

\textbf{Frecuencia de actualización: } Trimestral

\textbf{Problemas en la base: } - No tiene fechas especificas lo cual
limita un poco el analisis para ver en que periodos va la gente

\begin{itemize}
\tightlist
\item
  La fecha de actualizacion de la pagina es del 6 de octubre de 2021 y
  tiene datos del cuarto trimestre de 2022.
\end{itemize}

\textbf{Potencial de esta base: } - Tiene un monton de preguntas que se
hacen, desde cosas basicas como de donde vienen y a donde van hasta si
contrataron un paquete de viaje y que cosas incluye

\begin{itemize}
\item
  Se podria analizar los medios de transporte o porque estan haciendo
  este viaje
\item
  Tiene mucha informacion de que cosas hace en el viaje asi que se
  pueden hacer un monton de preguntas
\end{itemize}

\textbf{Breve descripción: }Es una base de datos que recopila de manera
trimestral las respuestas de las encuestas que la gente responde al
viajar

\textbf{Preguntas a hacer a la base: } - Que tanto se usa internet para
organizar el viaje (reservas, etc.)

\begin{itemize}
\item
  Cuales son los destinos mas visitados, cual es el motivo y que medios
  de transporte usan
\item
  Que influencio la decision de visitar el destino?
\item
  Que tan seguido visitan el destino
\item
  Que tipos de actividades se realizan en el destino
\item
  Como la paso en el destino, cuales son los mejores y peores destinos
  filtrando por motivo (mejores lugares religiosos, turisticos, etc.)
\item
  Mostrar la calidad promedio de en las provincias (se podria hacer un
  top mejores y peores):

  \begin{itemize}
  \item
    Gastronomia
  \item
    Alojamiento
  \item
    Informacion turistica
  \item
    Higiene
  \item
    Seguridad
  \item
    Transporte (filtrando por tipo de transporte)
  \end{itemize}
\item
  Tamaño de la familia del que viaja
\item
  Edad promedio de la gente que responde la encuesta en relacion al
  destino que visita.
\item
  Nivel educativo de lugar que se visita y de donde viene
\item
  Nivel educativo y cobertura de salud
\item
  Cobertura de salud y destino/origen
\item
  Si la gente que no va por motivo religioso realiza una actividad
  religiosa
\item
  Preparacion promedio que se le dedica para ir al destino
\item
  Preparacion promedio que se e dedica filtrado por motivo
\item
  Nivel educativo por tipo de alojamiento
\item
  Duracion estadia por destino
\item
  Duracion estadia por nivel educativo
\item
  Duracion estadia por estado laboral
\end{itemize}

\textbf{Pedido de información: } - Se podria pedir que den informacion
un poco mas detallada y que sea por dia/por semana para que no den
informacion que pueda identificar a nadie.

\begin{itemize}
\tightlist
\item
  Actualizar los datos para incluir el primer trimestre de 2023 para
  analizar los viajes post-pandemia
\end{itemize}

\newline

Conclusiones

\textbf{General}

\begin{itemize}
\item
  \textbf{Los 5 destinos mas grandes fueron:}

  \begin{itemize}
  \tightlist
  \item
    Se puede ver que Buenos Aires (no CABA ni GBA) es uno de los
    principales destinos turisticos
  \end{itemize}
\end{itemize}

\begin{longtable}[]{@{}lr@{}}
\toprule()
Destinos & Cantidad \\
\midrule()
\endhead
Buenos Aires (Resto) & 109590 \\
Córdoba & 76014 \\
Santa Fe & 26155 \\
Mendoza & 25070 \\
Tucumán & 20476 \\
\bottomrule()
\end{longtable}

\begin{itemize}
\item
  \textbf{El principal motivo de visita:}

  \begin{itemize}
  \tightlist
  \item
    Hay bastante viaje por razones de salud, seria interesante analizar
    despues el resultado de esos viajes
  \end{itemize}
\end{itemize}

\begin{longtable}[]{@{}lr@{}}
\toprule()
Motivos & Cantidad \\
\midrule()
\endhead
Esparcimiento, ocio, recreacion & 262059 \\
Visitas a familiares o amigos & 143769 \\
Trabajo, negocios, motivos profesionales & 11132 \\
Otros & 6422 \\
Razones de salud & 4888 \\
\bottomrule()
\end{longtable}

\begin{itemize}
\item
  \textbf{Cantidad de gente que respondio la encuesta a lo largo del
  tiempo}

  \begin{itemize}
  \tightlist
  \item
    Pareciera que hay una tendiente de bajada general del turismo, se
    necesitaria un analisis mas profundo para afirmarlo
  \end{itemize}
\end{itemize}

\begin{Shaded}
\begin{Highlighting}[]
\NormalTok{cantidad\_respuestas}
\end{Highlighting}
\end{Shaded}

\includegraphics{Tp2_files/figure-latex/unnamed-chunk-2-1.pdf}

\begin{itemize}
\tightlist
\item
  \textbf{\% de los destinos:}
\end{itemize}

\begin{Shaded}
\begin{Highlighting}[]
\NormalTok{porcentaje\_destinos}
\end{Highlighting}
\end{Shaded}

\includegraphics{Tp2_files/figure-latex/unnamed-chunk-3-1.pdf}

\begin{longtable}[]{@{}lr@{}}
\toprule()
descripcion & Porcentaje \\
\midrule()
\endhead
Buenos Aires (Resto) & 25.2043440 \\
Córdoba & 17.4822795 \\
Santa Fe & 6.0153264 \\
Mendoza & 5.7657898 \\
Tucumán & 4.7092266 \\
Salta & 4.5337461 \\
Entre Ríos & 3.9930452 \\
CABA & 3.6692226 \\
Río Negro & 2.7527219 \\
San Luis & 2.7373127 \\
Chubut & 2.4953657 \\
Corrientes & 2.3776121 \\
San Juan & 2.2771075 \\
Jujuy & 2.2623883 \\
Santiago del Estero & 2.0110118 \\
Catamarca & 1.9525950 \\
Neuquén & 1.8900383 \\
Misiones & 1.6996086 \\
Partidos del GBA (Pcia. Bs. As.) & 1.3394479 \\
La Rioja & 1.1897260 \\
Tierra del Fuego & 1.0512734 \\
Chaco & 0.7200913 \\
La Pampa & 0.6579946 \\
Santa Cruz & 0.6264863 \\
Formosa & 0.5862385 \\
\bottomrule()
\end{longtable}

\begin{itemize}
\item
  \textbf{Cual es el trimestre del año en donde hay mas viajes?}

  \begin{itemize}
  \tightlist
  \item
    El trimestre 1 es el que tiene mas viajes, con un total de 139567
    viajes
  \end{itemize}
\end{itemize}

\begin{Shaded}
\begin{Highlighting}[]
\NormalTok{porcentaje\_trimestres}
\end{Highlighting}
\end{Shaded}

\includegraphics{Tp2_files/figure-latex/unnamed-chunk-4-1.pdf}

\end{document}
